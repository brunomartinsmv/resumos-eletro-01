\newpage\section{ Deslocamento Elétrico}

O deslocamento elétrico é um conceito central em Eletromagnetismo. Ele descreve a densidade de carga elétrica livre  -  que é a única carga que nós conseguimos controlar - em um dado ponto de um material dielétrico. Quando aplicamos um campo elétrico externo a um material dielétrico, os elétrons das átomos deslocam-se parcialmente de suas posições de equilíbrio, criando uma polarização elétrica, ou seja, produzir acúmulos de carga ligada dentro do e na superfície do dielétrico. O deslocamento elétrico ($D$) é a soma do campo elétrico aplicado ($E$) e da polarização elétrica ($P$), onde
\begin{equation}
\vec{D} = \varepsilon_0 \vec{E} + \vec{P},
\end{equation}

onde $\varepsilon_0$ é a permissividade elétrica do vácuo. Como todas as equações de Maxwell são escritas em termos de fontes livres, então $\vec{D}$ na lei de Gauss se torna

\begin{equation}
    \nabla \cdot \vec{D} = \rho_l .
\end{equation}


\subsection{Dielétricos Lineares}

Dielétricos Lineares são caracterizados por não conduzirem corrente elétrica significativa, mas respondem de maneira linear quando expostos a um campo elétrico externo. À medida que o campo elétrico aumenta, a polarização elétrica do material também aumenta de forma proporcional. Essa relação é expressa por meio da susceptibilidade elétrica ($\chi_e$), uma constante do material que quantifica a capacidade do material de se polarizar sob a influência do campo elétrico.

A equação que descreve a relação entre o campo elétrico ($\vec{E}$), a polarização ($\vec{P}$) e a susceptibilidade elétrica é:
\begin{equation}
    \vec{P}=\epsilon_0 \chi_e \vec{E}, 
\end{equation}


onde $\varepsilon_0$ representa a permissividade elétrica do vácuo.

Quando um dielétrico linear é inserido em um campo elétrico, ele aumenta o deslocamento elétrico ($\vec{D}$) em relação ao campo aplicado ($\vec{E}$) devido à contribuição da polarização do material. Isso significa que o dielétrico efetivamente aumenta a capacidade de armazenamento de carga e energia elétrica, sendo amplamente utilizado em capacitores.

\subsection{ Energia em Sistemas Dielétricos}
Quando um dielétrico é inserido em um capacitor, a energia armazenada no capacitor aumenta devido ao aumento da capacitância resultante do dielétrico. A energia armazenada ($\vec{W}$) em um capacitor com dielétrico é dada pela fórmula
\begin{equation}
   \vec{W} = \frac{CV^2}{2},
\end{equation}
onde $C$ é a capacitância do capacitor e $V$ é a diferença de potencial entre as placas do capacitor.

Quando um capacitor é preenchido com um dielétrico linear, a capacitância é maior do que no vácuo. Isso significa que o capacitor pode armazenar uma quantidade maior de carga elétrica para o mesmo potencial elétrico aplicado. Portanto, a energia armazenada em um sistema dielétrico é maior devido à maior capacidade de armazenamento de carga, mantendo o mesmo potencial elétrico.
\newpage



