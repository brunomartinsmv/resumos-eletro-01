\newpage\section{Magnetostática}
Até agora sabíamos que uma carga estática gerava campo elétrico, mas não sabíamos sobre cargas em movimento. As cargas em movimento geram os \textbf{campos magnéticos} e lhe é atribuído a letra $\vec{B}$. 

Um campo magnético $\vec{B}$ é capaz de exercer uma força, conhecida como \textbf{força magnética} $(\vec{F}_{mag})$ que é a força exercida por um campo magnético em um objeto ou partícula carregada magneticamente que está em movimento. Essa força atua perpendicularmente à direção do movimento e à direção do campo magnético.

A força magnética é descrita pela Lei de Lorentz, que estabelece que a força magnética ($ \vec{F}_{mag}$) em uma partícula carregada ($q$) em movimento com velocidade ($\vec{v}$) em um campo magnético ($\vec{B}$) é dada pela fórmula:

\begin{equation} \label{fmag}
    \vec{F}_{mag} = q  (\vec{v} \times\vec{B}),
\end{equation}

    onde a direção dessa força pode ser encontrada utilizando a regra da mão direita.

Uma particularidade da equação \ref{fmag} é de que a força magnética $\vec{B}$ não realiza trabalho $dW_{mag}$ sobre cargas. O trabalho realizado é calculado multiplicando a componente $\vec{F}_{mag}$ pelo deslocamento $d\mathbf{l}$. Matematicamente, o trabalho infinitesimal ($dW$) realizado pela força magnética em um deslocamento dado por um vetor deslocamento ($d\mathbf{l} = \vec{v} \textit{dt}$) é dado por:

\begin{equation}
    dW_{mag} = F_{mag} \cdot d\textbf{l} = Q(\vec{v} \times \vec{B}) \cdot \vec{v} dt,
\end{equation}

mas o vetor $\Vec{v} \times \vec{B}$ sempre terá um ângulo de 90°$(\pi / 2)$ em relação ao vetor velocidade $\vec{v}$, com isso o produto escalar entre os dois vetores $(|A||B| \cos \theta)$ resultará em zero $(\cos \pi /2 = 0)$. Então o trabalho realizado por força magnética $dW_{mag}$ será:
\begin{equation}
    dW_{mag} = 0.
\end{equation}
\newpage