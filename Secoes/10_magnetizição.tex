\newpage\section{Magnetização e Materiais Magnéticos}A magnetização é um fenômeno em que um material adquire propriedades magnéticas devido à interação de seus átomos com um campo magnético externo. Quando um material é exposto a um campo magnético, seus dipolos magnéticos tendem a se alinhar em direção do campo aplicado, resultando na criação de um campo magnético líquido no material.

A magnetização pode ocorrer em diferentes materiais, e o resultado depende da orientação dos dipolos magnéticos em relação ao campo magnético. Esses resultados são observados em três categorias principais de materiais: \textbf{paramagnéticos}, \textbf{diamagnéticos} e \textbf{ferromagnéticos}.



Nos materiais \textbf{paramagnéticos}, os momentos magnéticos dos átomos se alinham \textit{paralelamente} com o campo magnético externo. Quando o campo magnético \textbf{$\vec{B}$} é removido, os momentos magnéticos retornam à sua orientação aleatória original. 

Os materiais \textbf{diamagnéticos} possuem momentos magnéticos \textit{opostos} ao campo magnético aplicado e também perdem quando o campo magnético \textbf{$\vec{B}$} é removido. 

Já materiais \textbf{ferromagnéticos} tem os seus dipolos magnéticos apontado paralelamente ao campo magnético \textbf{$\vec{B}$}. O que os diferencia dos outros materiais é a capacidade de reter a direção do dipolo magnético mesmo após o campo magnético ser removido. Isso os torna muito úteis para estudos e aplicações práticas, permitindo a criação de ímãs permanentes e a exploração de diversos fenômenos magnéticos.

