\newpage
\section{Cálculo Diferencial}
A derivada ordinária de uma função $f(x) = t$ representa o quanto que aquela função variou em relação a  uma variação infinitesimal de t, como entender a velocidade de uma função. Matematicamente temos que
\begin{equation}
    dv = \frac{dx}{dt} .
\end{equation}



A derivada parcial é a derivada para uma função com duas ou mais variáveis. Neste caso há de se derivar a função em relação a uma das variáveis e manter as outras constantes. Então a função $z = f(x,y)=x^2 + 2y^3$ terá como derivada parcial em $x$ 
\begin{equation}
    \frac{\partial z}{\partial x} = 2x .
\end{equation}
O operador \textit{del} $\nabla$ (conhecido como Nabla) é importante para a construção do gradiente, divergente e rotacional, e é escrito da seguinte maneira para um sistema de três coordenadas: 

\begin{equation}
    \nabla \Vec{V} = \frac{\partial}{\partial x} \hat{\textbf{x}} + \frac{\partial}{\partial y} \hat{\textbf{y}}+\frac{\partial}{\partial z} \hat{\textbf{z}}.
\end{equation}

O Nabla é um operador mas tem uma flecha em cima, isso acontece porque ele é um operador vetorial, então ele sozinho não tem significado, mas ao colocar com uma função terá um significado.

Em uma função com várias variáveis podemos fazer algumas operações usando as derivadas parciais. O gradiente de uma função vai mostrar como que a mesma vai funcionar quando uma das variáveis seja alterada. 
O Gradiente de uma função que contém as derivadas parciais formando um vetor ao aplicar em um campo escalar. Então em uma função $f(x,y,z) = x^2 + 2xy^2 + 3z$, o seu gradiente será:

\begin{equation}
    \nabla F = 2x \hat{\textbf{x}} + 4y \hat{\textbf{y}} + 3 \hat{\textbf{z}} .
\end{equation}

O gradiente tem direção (onde terá a maior velocidade), magnitude (inclinação da reta naquela direção) e sentido (para onde aponta). Quando o gradiente foi igual a zero teremos um ponto crítico, ou no caso de uma montanha (exercício $1.12$ do Griffiths), o topo será quando igualarmos o gradiente a zero. 


O divergente é um campo escalar multiplicado pelo operador \textit{del} 

\begin{equation*}
    \nabla \cdot \textbf{v} = (\hat{\textbf{x}} \frac{\partial}{\partial x} + \hat{\hat{\textbf{y}}} \frac{\partial}{\partial y} + \hat{\hat{\textbf{z}}} \frac{\partial}{\partial z}) \cdot (v_{x} {\hat{\textbf{x}}} + v_{y} {\hat{\textbf{y}}} + v_{z} {\hat{\textbf{z}}}) ,
\end{equation*}


\begin{equation} \label{divergente}
    \nabla \cdot \textbf{v} = \frac{\partial v_{x}}{\partial x} + \frac{\partial v_{y}}{\partial y} + \frac{\partial v_{z}}{\partial z} ,
\end{equation}
\myequations{Divergente de um campo escalar}

e vai indicar o quanto que o vetor vai estar divergindo do ponto em questão. 

Se \ref{divergente} é igual a $0$ então não há divergência,  se \ref{divergente} for menor do que zero então o campo tende a ficar mais denso dentro do ponto em questão, caso \ref{divergente} seja positivo então há menos densidade naquele ponto do campo. Esta ideia pode ser entendida como uma torneira, onde

\begin{itemize}
    \item Torneira aberta: Divergente $ > 0$;
    \item Torneira fechada: Divergente $= 0$.
\end{itemize}

Em um caso hipotético onde a água entre na torneira o divergente seria $< 0$. 

O rotacional é uma função vetorial dado por um campo vetorial multiplicado por \textit{del}: 
A matriz contendo as derivadas parciais em relação às coordenadas $x$, $y$ e $z$, juntamente com os versores $\hat{x}$, $\hat{y}$ e $\hat{z}$, é dada por:
\begin{equation*}
\nabla \times  V = 
\begin{bmatrix}
{\hat{\textbf{x}}} & {\hat{\textbf{y}}} & {\hat{\textbf{z}}} \\
\frac{\partial}{\partial x} & \frac{\partial}{\partial y} & \frac{\partial}{\partial z} \\
v_{x} & v_{y} & v_{z}

\end{bmatrix}
\end{equation*}

\begin{equation}
    \nabla \times  V = \hat{\textbf{x}} \left(\frac{\partial v_{z}}{\partial y} - \frac{\partial v_{y}}{\partial z }\right) + \hat{\textbf{y}} \left(\frac{\partial v_x}{\partial z } - \frac{\partial v_{z}}{\partial x}\right) + \hat{\textbf{z}} \left(\frac{\partial v_{y}}{\partial x} - \frac{\partial v_{x}}{\partial y}\right) . 
\end{equation}
\myequations{Rotacional de um campo vetorial}

Essa função vai medir o quanto que o vetor rotaciona em relação a um ponto em discussão, impactando um vetor tridimensional próxima que irá rotacionar e assim sucessivamente. 
