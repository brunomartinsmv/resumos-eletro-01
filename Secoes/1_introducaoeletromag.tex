\section{Introdução ao Eletromagnetismo}
O eletromagnetismo é um campo de estudo que se concentra nos efeitos do campo e da radiação produzidos pelo movimento de cargas elétricas. Ele fornece uma descrição matemática e física da interação entre essas cargas, visando compreender o comportamento de uma carga elétrica quando exposta a campos elétricos e magnéticos gerados por outras cargas.

A matéria é o resultado da combinação do estudo da eletricidade com o magnetismo. Essa união se deve às descobertas feitas por Ørsted em 1820, quando ele observou que uma corrente elétrica gerava um campo magnético, e por Faraday em 1831, quando ele descobriu que um campo magnético podia induzir uma corrente elétrica. Essas descobertas foram fundamentais para o desenvolvimento do eletromagnetismo como campo de estudo.

O eletromagnetismo está inserido nos quatro principais domínios da física: mecânica clássica, mecânica quântica, relatividade e teoria de campos. Ao considerarmos que um sistema reage quando uma força é aplicada sobre ele, podemos identificar as forças fundamentais envolvidas nesse processo.



As quatro principais forças fundamentais conhecidas são:

\begin{enumerate}
    \item  Força gravitacional: é descrita pela teoria da gravitação de Albert Einstein e Isaac Newton, chamada de relatividade geral. Essa teoria descreve a interação gravitacional entre corpos massivos.

    \item Força eletromagnética: é descrita pela teoria eletromagnética de James Maxwell. Ela engloba tanto a interação elétrica quanto a interação magnética entre partículas carregadas.

    \item Força nuclear forte: é responsável pela coesão do núcleo atômico e é descrita pela teoria cromodinâmica quântica (QCD). Essa teoria descreve a interação entre quarks e glúons, que são as partículas constituintes dos hádrons.

    \item Força nuclear fraca: é responsável por processos de decaimento radioativo e é descrita pela teoria eletrofraca. Essa teoria unifica a interação fraca com a interação eletromagnética em um único formalismo.
\end{enumerate}
\newpage
Onde essas forças podem ser ranqueadas por meio de suas intensidades

\begin{table}[h]
\centering
\caption{Forças fundamentais na natureza por ordem de intensidade}
\label{tab:forcasnatureza}
\begin{tabular}{|c|c|}
\hline
\textbf{Força}  & \textbf{Intensidade}    \\ \hline
Forte           & 10                      \\ \hline
Eletromagnética & $10^{-2}$  \\ \hline
Fraca           & $10^{-13}$\\ \hline
Gravitacional   & $10^{-42}$\\ \hline
\end{tabular}
\end{table}



Todas as forças são explicadas teoricamente com base no eletromagnetismo, pois é a única teoria totalmente compreendida. Ao estudar o efeito do campo elétrico, é fundamental compreender as propriedades das cargas elétricas. As cargas elétricas podem ser classificadas em negativas e positivas. Essa distinção é baseada em observações experimentais e descreve a natureza das partículas carregadas. Por exemplo, elétrons têm carga negativa, enquanto prótons têm carga positiva.
    
A carga elétrica é sempre conservada, a menos que haja contato com uma carga oposta equivalente, por exemplo, se um objeto carregado positivamente tocar um objeto carregado negativamente, elétrons do objeto negativo podem se mover para o objeto positivo, neutralizando parte ou toda a carga positiva.  A quantidade total de cargas positivas e negativas permanece a mesma, mas sua distribuição pode ser alterada pela atração e repulsão entre as cargas. Além disso, a carga elétrica é quantizada, o que significa que ela possui um valor mínimo e não pode ser dividida infinitamente.
