\newpage\section{Polarização}
Na natureza temos dois tipos de materiais:
\begin{enumerate}
    \item \textbf{Condutores}:  São materiais que possuem a capacidade de permitir o fluxo de corrente elétrica com relativa facilidade. Isso ocorre porque os elétrons em um condutor estão fracamente ligados aos seus átomos, permitindo que eles se movam livremente sob a influência de uma diferença de potencial elétrico.
    \item \textbf{Isolantes} (dielétricos): São materiais que não conduzem eletricidade facilmente. Quando um campo elétrico é aplicado a um dielétrico, os elétrons não podem se movimentar livremente através dele, como ocorre em um condutor. Isso acontece devido à forte ligação dos elétrons em um dielétrico aos seus átomos, impedindo o fluxo de corrente elétrica.
\end{enumerate}

Ao aplicarmos um campo elétrico $\vec{E}$ a um condutor, as partículas se movem sem impedimento. No entanto, quando aplicamos o mesmo campo a um dielétrico, ocorre uma redistribuição das cargas. O núcleo se desloca em direção ao campo elétrico, enquanto os elétrons se movem em direção oposta, criando um dipolo de polarização $\vec{p}$. Esse dipolo é diretamente proporcional ao campo elétrico aplicado. Se o campo $\vec{E}$ for excessivamente intenso, o isolante pode se tornar ionizado.

No caso das moléculas, nem sempre ocorre o mesmo efeito, uma vez que moléculas polares são aquelas em que os elétrons estão distribuídos de maneira assimétrica, resultando em uma carga positiva parcial em uma extremidade da molécula e uma carga negativa parcial na outra extremidade. Esse desequilíbrio de cargas cria um dipolo elétrico na molécula, com uma separação de carga.

Quando um campo elétrico externo é aplicado a um material que contém moléculas polares, essas moléculas tendem a se alinhar na direção do campo. Esse alinhamento é conhecido como polarização. O campo elétrico externo exerce forças sobre as cargas parciais nas moléculas polares, fazendo com que elas girem ou se orientem de acordo com a direção do campo.

