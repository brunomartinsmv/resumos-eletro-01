\newpage\section{Ferromagnetismo}O \textbf{ferromagnetismo} é um fenômeno magnético em que certos materiais, como ferro,  se magnetizam de forma permanente e retem o seu magnetismo mesmo após a retirada deste campo magnético.

Uma característica distintiva do ferromagnetismo é a presença de domínios magnéticos em um material ferromagnético. Os domínios são regiões microscópicas no material onde os momentos magnéticos dos átomos estão alinhados em uma mesma direção. Cada domínio atua como um pequeno ímã, contribuindo para a magnetização total do material.


Em um material não magnetizado, os domínios magnéticos estão distribuídos aleatoriamente, resultando em um cancelamento das forças magnéticas entre eles. Portanto, a magnetização total do material é próxima de zero.

No entanto, quando um campo magnético externo é aplicado ao material ferromagnético, os domínios tendem a se alinhar com a direção do campo. À medida que o campo magnético se fortalece, os domínios se alinham cada vez mais, aumentando a magnetização total do material.

Quando o campo magnético é removido, alguns dos domínios podem permanecer alinhados, mantendo a magnetização do material. Essa propriedade é conhecida como remanência. No entanto, a magnetização remanescente pode ser reduzida ou eliminada por meio de forças externas, como aquecimento intenso ou aplicação de campos magnéticos opostos. Se o campo magnético externo for variado, obtemos o que é chamado de \textbf{ciclo de histerese}. Esse ciclo descreve o comportamento da magnetização de um material ferromagnético à medida que o campo magnético externo é variado.

Ao aumentar gradualmente a intensidade do campo magnético aplicado a um material ferromagnético a partir de zero, a magnetização do material aumenta. No entanto, a relação entre a intensidade do campo magnético aplicado e a magnetização não é linear. Em vez disso, ela exibe uma relação não linear e complexa.

Durante o aumento do campo magnético, o material passa por um ponto conhecido como ponto de saturação. Nesse ponto, a magnetização atinge seu valor máximo e não pode mais aumentar, mesmo que o campo magnético seja aumentado ainda mais. A partir desse ponto, o material está completamente magnetizado.

Agora, se começarmos a reduzir gradualmente a intensidade do campo magnético externo, o material não retorna à sua magnetização original. Isso ocorre porque o material possui uma propriedade chamada coercividade, que é a resistência do material a ter sua magnetização revertida. Para reverter a magnetização, é necessário aplicar um campo magnético oposto, com intensidade suficiente para superar a coercividade do material.

O processo de redução do campo magnético até o ponto em que a magnetização do material se torna zero é chamado de ciclo de desmagnetização. Nesse ponto, o material retorna ao estado não magnetizado.

Agora, se continuarmos a aumentar novamente a intensidade do campo magnético, observaremos um comportamento interessante. O material não segue a mesma curva de magnetização do primeiro ciclo, mas segue uma curva de magnetização diferente. Essa diferença ocorre porque o material apresenta uma memória magnética. Os domínios magnéticos que não foram completamente desfeitos durante o ciclo de desmagnetização influenciam a resposta magnética do material durante o ciclo subsequente. Essa memória resulta na formação de um novo conjunto de domínios magnéticos durante o processo de remagnetização.

O ciclo completo de magnetização, desmagnetização e remagnetização de um material ferromagnético é conhecido como \textbf{ciclo de histerese magnética.} Esse fenômeno é frequentemente representado graficamente em um gráfico que mostra a magnetização em função do campo magnético aplicado. O ciclo de histerese mostra a diferença entre a magnetização do material quando o campo magnético está aumentando e quando está diminuindo.


    %\includegraphics[width=0.5\textwidth]{Screenshot 2023-05-14 190347.png} 


A temperatura também desempenha um papel importante no ferromagnetismo. Acima de uma determinada temperatura crítica, conhecida como \textbf{temperatura de Curie}, o material ferromagnético perde suas propriedades magnéticas. Nesse ponto, a energia térmica é suficiente para perturbar a ordem dos domínios magnéticos, causando sua desorganização e redução da magnetização. Abaixo da temperatura de Curie, o material recupera suas propriedades ferromagnéticas.

